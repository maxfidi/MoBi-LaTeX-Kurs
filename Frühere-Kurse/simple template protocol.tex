% simple template for protocols
% created by Lena Meßner
% last update: 14.06.2022


% !TeX spellcheck = en_GB
\documentclass[a4paper,12pt]{report}


\usepackage[T1]{fontenc}
\usepackage[utf8]{inputenc}
\usepackage[greek, ngerman, main=english]{babel}
\usepackage{microtype}
\usepackage{titlesec}
\usepackage{geometry}
\usepackage{setspace}
\usepackage{parskip}
\usepackage{natbib}
\usepackage{caption}
\usepackage{blindtext}
\usepackage{graphicx}
\usepackage{booktabs}
\usepackage{multirow}
\usepackage{amsmath}
\usepackage{amsfonts}
\usepackage{amssymb}
\usepackage[hyphens]{url}
\usepackage{listings}
\usepackage{placeins}
\usepackage{pdfpages}
\usepackage{icomma}
\usepackage{pdflscape}
\usepackage{gensymb}
\usepackage{cmll}
\usepackage{color}
\usepackage{xcolor}
\usepackage[printonlyreused]{acronym}
\usepackage{cell}
\usepackage{array}
\usepackage{varwidth}
\usepackage{subcaption}  % statt subfig



%\newcommand{\changefont}[3]{\fontfamily{#1}\fontseries{#2}\fontshape{#3}\selectfont} 
%für andere Schriftart (unwichtig!)

\titleformat{\chapter}
[hang]
{\Large}
{\thechapter}
{10pt}
{\Large}
[{\titlerule[0.3pt]}]


\titlespacing*{\chapter}{0pt}{0pt}{15pt}[0pt] %{Einrücken}{Abstand oben}{Abstand unten}[Rand]


\titleformat{\section}
{\large}
{\thesection}
{5pt}
{\large}

\titlespacing*{\section}{0pt}{17pt}{8pt}[0pt]


\DeclareCaptionStyle{Captions}{labelfont={small, bf}, textfont={small}, aboveskip=0.3cm, belowskip=0.5cm}
\DeclareCaptionStyle{Ausnahme}{labelfont={small, bf}, textfont={small}, aboveskip=0.3cm, belowskip=0cm}

\captionsetup{style=Captions}      %captions sind die Bildbeschriftungen!!


\geometry{outer=25mm,
	inner=25mm,
	top=25mm,
	bottom=25mm}

\onehalfspacing

\renewcommand{\arraystretch}{1.1}  % definiert vertikalen Abstand der Tabellenzeilen
	

\begin{document}
\onehalfspacing
\pagenumbering{roman}
	
\begin{titlepage}
	\begin{flushleft}University of Heidelberg\\
	 Faculty of Biosciences\\
	 Molecular Biotechnology Bachelor Program\\
	\end{flushleft}
\vspace*{6.5cm}
	\begin{center}
		\huge AMPAR-TARP interaction\\ \bigskip
    	\Large Lab Course II Bioinformatics\\ \smallskip
    	\large 06.09.2021 - 24.09.2021
    	\normalsize
	\end{center}
\vspace*{\fill}
	\begin{flushright} 
	\large Lena Meßner\\ \normalsize
	matriculation number 1234567
	\end{flushright}
\end{titlepage}

\renewcommand\abstractname{} % Abstract soll keinen Titel haben
\begin{abstract}
	
\end{abstract}

{\setlength{\parskip}{0.2cm}		% Kleinerer Zeilenabstand zwischen den Einträgen
	%	Abkürzungsverzeichnis
	\chapter*{List of Abbreviations}
	\addcontentsline{toc}{chapter}{List of Abbreviations}
	\begin{acronym}[ABCDEFGHIJK]	% Länge des Strings in den [..] entspricht dem Abstand der zwischen Acronym und Vollname gelassen wird
		% => mindestens so lange wie das längste Acronym
		% Abkürzungen
		%\acro{acronym}{full name} -> hier werden die erstellt, im Text dann mit \ac{acronym} benutzen!
		%list of abbreviation lists abbreviations in the order they are generated here!
		\acro{TARP}{transmembrane AMPAR regulatroy protein}
		\acro{AMPAR}{$\alpha$-amino-3-hydroxy-5-methyl-4-isoxazolepropionic acid receptor}
		\acro{iGluR}{ionotropic glutamate receptor}
		\acro{cryo-EM}{cryo-electron microscopy}
		
		
		% Formelzeichen
		%\acro{acronym}[$ Formelzeichen $]{full name}
		
		% als benutzt markierte Acronyme	
		%\acused{acronym}
		
	\end{acronym}

\tableofcontents
\addcontentsline{toc}{chapter}{Contents}

\newpage


\onehalfspacing
\pagenumbering{arabic}
\chapter{Introduction}
\blindtext





\chapter{Materials and Methods}





\chapter{Results}


%\begin{figure}[h!]
%	\centering
%	\subfloat[Hellfeldfoto]{\includegraphics[height=3.55cm]{hellfeld.png}\label{hellfeld}}
%	\qquad
%	\subfloat[Fluoreszenzfoto]{\includegraphics[height=3.55cm]{fluoreszenz.png}\label{fluoreszenz}}
%	\qquad
%	\subfloat[Overlay]{\includegraphics[height=3.55cm]{overlay.png}\label{overlay}}
%	\caption[Mikroskopie]{In dieser Abbildung sind die bei der Mikroskopie erhaltenen Bilder nebeneinandergestellt: Das Bild aus dem Lichtmikroskop (a), das Fluoreszenzmikroskopbild (b) sowie das Overlay aus beiden Bildern (c). \label{mikroskopie}}
%\end{figure}


%\begin{figure}[h!]
%	\centering
%	\includegraphics[width=1.0\textwidth]{kalibrierethanol.png}
%	\caption[Kalibriergerade Ethanol]{Die Abbildung zeigt die aus den Messwerten erstellte Kalibriergerade: Die Ethanolkonzentration wurde gegen die gemessene Absorption aufgetragen und eine lineare Regressionsgerade wurde erstellt. \label{kalibrierethanol}}
%\end{figure}


\begin{equation*}
	f(x) = 0,264x - 0,0057
\end{equation*}

\begin{table}[h!]
	\centering
	\caption [Reaktionsraten]{In dieser Tabelle sind die berechneten Reaktionsraten mit der zugehörigen Konzentration an G6P dargestellt.\label{TabelleReaktionsraten}}
	
	\begin{tabular}{lp{1.5cm}l}
		\toprule
		c(G6P) [mM]	& &	Reaktionsrate [mM NADPH/s]	\\
		
		\midrule
		0	& &	(2,5 $\pm$ 0,7) $\cdot$ 10\textsuperscript{-5}	\\
		0,125	& &	(8,1 $\pm$ 0,3) $\cdot$ 10\textsuperscript{-5}	\\
		0,25	& &	(1,51 $\pm$ 0,18) $\cdot$ 10\textsuperscript{-4}	\\
		0,5	& &	(1,93 $\pm$ 0,12) $\cdot$ 10\textsuperscript{-4}	\\
		1	& &	(2,036 $\pm$ 0,003) $\cdot$ 10\textsuperscript{-4}	\\
		2	& &	(2,16 $\pm$ 0,06) $\cdot$ 10\textsuperscript{-4}	\\
		3	& &	(2,33 $\pm$ 0,07) $\cdot$ 10\textsuperscript{-4}	\\
		5	& &	(2,27 $\pm$ 0,09) $\cdot$ 10\textsuperscript{-4}	\\
		
		\bottomrule
	\end{tabular}
\end{table}


\FloatBarrier



		
\begin{align*}
f(x) &= 983,94x + 3796,8 = 0 \\
x &= -3,858771876
\end{align*}


\chapter{Discussion}

\appendix
\setcounter{chapter}{18}
\chapter{Supplemental Material}



%Literaturverzeichnis
\renewcommand{\bibname}{References}
\bibliographystyle{cell}
\bibliography{Literatur.bib}
\addcontentsline{toc}{chapter}{References}

%\titleformat{\chapter}
%[hang]
%{\Large}
%{}
%{0pt}
%{\Large}
%[{\titlerule[0.3pt]}]

\end{document}