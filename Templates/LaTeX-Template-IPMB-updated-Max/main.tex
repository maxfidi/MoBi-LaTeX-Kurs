\documentclass[parskip]{scrartcl}

% !TEX root='./Beispielprojekt.tex'

% Sprachen
\usepackage[LGR,T1]{fontenc}
\usepackage{polyglossia}
\setmainlanguage{german}
\newcommand{\textgreek}[1]{\begingroup\fontencoding{LGR}\selectfont#1\endgroup}

% Textstyle
\sffamily
\usepackage[margin=2.5cm]{geometry}

% Bilder
\usepackage{graphicx}

% Informationen
\author{\me}
\date{\today}

% Tabelle
\usepackage{tabu}
\usepackage{array}
\usepackage{booktabs}
\usepackage{enumitem}
\usepackage{mathtools}

% captions
\usepackage[font=sf, format=plain, labelfont=bf]{caption} % Nicer captions with different font/typefaces

% Mathematik
\usepackage{amsmath}
\usepackage{amssymb}
\usepackage{esint}
\newcommand*{\QED}{\null\nobreak\hfill\ensuremath{\square}}

% Equation numbering
\numberwithin{equation}{section}

% Biologie
\usepackage{dnaseq} 	% Setzen von DNA und Proteinsequenzen

% Chemie
\usepackage{mhchem}
\usepackage{siunitx}
\sisetup{
	per-mode=fraction,
	output-decimal-marker={,},
	separate-uncertainty,
	table-align-uncertainty=true
	}

% Bilder
\usepackage{graphicx}

%references
\usepackage[hidelinks]{hyperref}
\usepackage[noabbrev]{cleveref}
\usepackage[dvipsnames]{xcolor}
\hypersetup{
	colorlinks=true,
	linkcolor=black,
	citecolor=MidnightBlue,
	filecolor=MidnightBlue,
	urlcolor=MidnightBlue
}
\usepackage{soul}
\newcommand{\myul}[2][LimeGreen]{\setulcolor{#1}\ul{#2}\setulcolor{black}}

% Blindtext
\usepackage{blindtext}
\usepackage{lipsum}

% Kopf- und Fußzeile
\usepackage{scrlayer-scrpage,lastpage}
\setkomafont{pageheadfoot}{\normalsize \texttt}
\lohead{\today}
\rohead{\me}
\cfoot*{\thepage{}/\pageref{LastPage}}
\rofoot{\aufgabe}

%Literatur
\usepackage[style=alphabetic, backend=bibtex]{biblatex}
\addbibresource{literatur.bib}
\usepackage{csquotes}

% Arbeiten mit PDF Dateien
\usepackage{pdfpages}

\usepackage{easyReview}


% Title Page
% Definitionen für die Titelseite
% Change the content of the second bracket pair to what you need
\newcommand{\workauthor}{}
\newcommand{\worktitle}{}
\newcommand{\studentid}{}
\newcommand{\email}{}
\newcommand{\workyear}{}
\newcommand{\supervisor}{}

\sisetup{
        per-mode=fraction,
        output-decimal-marker={,},
        separate-uncertainty,
        table-align-uncertainty=true
        }
\begin{document}

\begin{titlepage}
    % HIER NICHTS ÄNDERN SONDER BEI DEN DEFINITIONEN OBEN DRÜBER    
    {\usekomafont{title}
    {\vspace*{1cm}\Huge Unterüberschrift einsetzen}
    
    {\vspace{1cm}\Huge\worktitle{}}}
    
    \vspace{\stretch{2}}
    
    {\Large \workauthor{}}
    
    Matrikelnummer: \studentid{}
    
    E-Mail Adressen: \email{}
    
    \vspace{\stretch{6}}
    
    Essay Seminar für das \workyear{} Fachsemester
    
    Supervisor: \textbf{\supervisor{}}
    
    \vspace{\stretch{4}}
    
  
    
\end{titlepage}


%    Abkürzungsverzeichnis
{\setlength{\parskip}{0.2cm}
\section*{Abbreviations}
    \begin{acronym}[LC-MS/MS23]
        % A B C D E F G H I J K L M N O P Q R S T U V W X Y Z        
        % Abkürzungen
        \acro{Amp}{Ampicillin}
        % Formelzeichen
        
        
        % als benutzt markierte Acronyme    
        
        
    \end{acronym}
}
\tableofcontents
\newpage
Das ist ein Beispielsatz mit Zitation \citep{Alberts.2015}.

\input{chapter/introduction}
%! TEX root=../main.tex
\section{Materials and Methods}

\input{chapter/results}
%! TEX root=../main.tex
\section{Discussion}


\bibliography{bib/essay.bib}

\end{document}

